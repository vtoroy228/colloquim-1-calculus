\documentclass[12pt, a4paper]{article}

\usepackage{header}
\author{made by Белянко Николай БПМИ257 \\
\href{https://t.me/nikbel26}{по всем вопросам}, \href{https://github.com/LoDThe/hse-tex}{исходники}}
\title{Математический анализ коллоквиум 1}
\begin{document}
    \maketitle
    \tableofcontents
    \newpage
    \section{Вопросы на определния и формулировки}
    \subsection{Сформулируйте аксиому непрерывности для вещественных чисел.}
        Если $A, B \subset \RR$ и множество $A$ лежит слева от множества $B$ (т.е. $\ \forall a \in A \ \forall b \in B \ a \leq b$), то $\exists \ c \in \RR : \forall a \in A \ \forall b \in B \ a \leq c \leq b$
    \subsection{Сформулируйте определение верхней и нижней грани, а также максимума множества.}
    \subsubsection{Верхняя и нижняя грань. [2.9]}
    \begin{enumerate}
        \item Число $M \in \RR$ называется верхней гранью множества $A \subset \RR$, Если число $M$ лежит справа от множества $A$, т.е. $\forall a \in A \ \ A \leq M$.
        \item Множество $A \subset \RR$ называется ограниченным сверху, если существует (конечная) верхняя грань этого множества: $\exists M \in \RR : \forall a \in A \ \ A \leq M$.
        \item Число $m \in \RR$ называется нижней гранью множества $A \subset \RR$, если число $m$ лежит слева от множества, т.е. $\forall a \in A \ \ a \geq m$.
        \item Множество $A \subset \RR$ называется ограниченным снизу, если существует (конечная) нижняя грань этого множества: $\exists m \in \RR : \forall a \in A \ \ a \geq m$.
        \item Множество $A$ называется ограниченным, если $A$ огранчиено сверху и ограничено снизу.
    \end{enumerate}
    \subsubsection{Максимум и минимум множества. [2.10]}
    Число $M$ называется максимальным элементом множества $A \subset \RR (M = \max A)$, если
    \begin{itemize}
        \item $M \in A$,
        \item $M$ является верхней гранью $A$.
    \end{itemize}
    Число $m$ называется минимальным элементом множества $A \subset \RR (m = \min A)$, если
    \begin{itemize}
        \item $m \in A$,
        \item $m$ является нижней гранью $A$.
    \end{itemize}
    \subsection{Сформулируйте определение точной верхней и нижней грани.}
    \subsubsection{Точная верхняя грань (супремум). [2.12]}
    Число $M \in \RR$ называется точной верхней гранью или супремумом множества $A \subset \RR (M = \sup A)$, если 
    \begin{enumerate}
        \item $M$ является верхней гранью множества $A$ и
        \item не существует числа, меньшего, чем $M$, и являющегося верхней гранью множества $A$, то есть
    \end{enumerate}
    \begin{itemize}
        \item  $\forall a \in A \ \ a\leq M$ и
        \item $\lnot (\exists M' \in \RR : M' < M \ \text{и} \ \forall a \in A \ \ a \leq M')$.
    \end{itemize}
    \subsubsection{Точная нижняя грань (инфинум). [2.14]}
    Число $m \in \RR$ называется точной нижней гранью или инфинумом множества $A \subset \RR (m = \inf A)$, если $m$ является максимальной нижней гранью $A$.
    \subsection{Сформулируйте определение числовой последовательности. [3.1]}
    Числовой последовательностью $\{a_n\}$ называется функция $a \colon \NN \to \RR$, где $a(n) = a_n$ для любого $n \in \NN$. Элемент последовательности --- это пара $(n, a_n)$, где $n$ --- номер элемента последовательности, а $a_n$ --- значение элемента последовательности. 
    \begin{remark}
        Последовательностью элементов из множества $X$ называется функция $f    \colon \NN \to X$, которая каждому натуральному числу ставит в соотвествие элемент из множества $X$. Последовательности обычно записывают как $a_1, a_2, a_3, \dots $, где $a_n = f(n)$
    \end{remark}
    \subsection{Приведите определение предела числовой последовательности. [3.3]}
    Число $a \in \RR$ называется пределом последовательности $\{a_n\}$ (пишут $a =  \lim_{n \to \infty} a_n$ или $a_n \to a$ при $n \to \infty$), если
    \begin{equation*}
        \forall \eps > 0 \ \ \exists N(\eps) : \ \ \forall n \geq N(\eps) \ \ a_n \in B_{\eps}(a), 
    \end{equation*}
    т.е.
    \begin{equation*}
        \forall \eps > 0 \ \ \exists N: \forall n \geq N |a_n - a| < \eps
    \end{equation*}
    Заметим, что в формулах для каждого $\eps > 0$ существует свое число $N$, то есть $N$ зависит от $\eps$. Чтобы подчеркнуть эту зависимость иногда вторую формулу переписывают в виде: 
    \begin{equation*}
        \forall \eps > 0 \ \ \exists N = N(\eps): \ \ \forall n \geq N \ \ |a_n - a| < \eps
    \end{equation*}
    \subsection{Сформулируйте лемму об отделимости для сходящейся числовой последовательности. [3.10]}
    Если $a_n \to a$ и $a \ne 0$, то найдется номер $N \in \NN$, для которого $|a_n| > \frac{|a|}{2} > 0$ при $n > N$

    \begin{proof} Действительно, взяв $\eps = \frac{|a|}{2}$ в определении сходимости последовательности к числу $a$, получаем номер $N \in \NN$, для которого $|a_n - a| < \frac{|a|}{2}$ при $n > N$. Тогда, при $n > N$, выполнено $|a| - |a_n| \leq |a_n - a| < \frac{|a|}{2}$, что равносильно доказываемому утверждению. \end{proof}
    \subsection{Перечислите арифметические свойства предела последовательности. [3.11]}
    Пусть $ \lim_{n \to \infty} a_n = a$ и $ \lim_{n \to \infty} b_n = b$. Тогда
    \begin{enumerate}
        \item $ \lim_{n \to \infty} (\alpha a_n + \beta b_n) = \alpha a + \beta b \ \ \forall \alpha, \beta \in \RR$;
        \item $ \lim_{n \to \infty} a_n b_n = ab$;
        \item если $b \ne 0, b_n \ne 0$, то $ \lim_{n \to \infty} \frac{a_n}{b_n} = \frac{a}{b}$.
    \end{enumerate}
    \subsection{Сформулируйте теорему о переходе к пределу в неравенстве. [3.12]}
    Пусть $ \lim_{n \to \infty} a_n = a,  \lim_{n \to \infty} b_n = b$. Если для некоторого номера $N$ выполнено $a_n \leq b_n$ при $n > N$, то $a \leq b$.
    \subsection{Приведите формулировку леммы о зажатой последовательности. [3.11]}
    Пусть $ \lim_{n \to \infty} a_n =  \lim_{n \to \infty} b_n = a$ и для некоторого $N \in \NN$ выполнено $a_n \leq c_n \leq b_n$ при $n > N$. Тогда $ \lim_{n \to \infty} c_n = a$
    \subsection{Приведите формулировку теоремы Вейерштрасса о пределе последовательности. [3.14]} 
    Пусть последовательность $\{a_n\}^{\infty}_{n=1}$ не убывает $(a_n \leq a_{n + 1})$ и ограничена сверху. Тогда эта последовательность сходится к своему супремуму.

    Аналогично, пусть последовательность $\{a_n\}^{\infty}_{n=1}$ не возрастает $(a_{n+1} \leq a_n)$ и ограничена снизу. Тогда эта последовательность сходится к своему инфимуму.
    \subsection{Определите число $e$. [3.17]}
    Предел последовательности $a_n = \left(1 + \frac{1}{n}\right)^n$ называют \textbf{числом Эйлера} и обозначают
    \begin{equation*}
         \lim_{n \to \infty} \left(1 + \frac{1}{n}\right)^n = e
    \end{equation*}
    \subsection{Сформулируйте принцип вложенных отрезков. [3.18]}
    Всякая последовательность $\{[a_n, b_n]\}^{\infty}_{n=1}$ вложенных отрезков (т.е. $[a_{n + 1}, b_{n + 1}] \subset [a_n, b_n]$) имеет общую точку. Кроме того, если длины отрезков стремятся к нулю, т.е. $b_n - a_n \to 0$, то такая общая точка только одна.
    \subsection{Что такое подпоследовательность и частичный предел? [3.19]}
    Пусть задана последовательность $\{a_n\}^{\infty}_{n=1}$ и пусть задана возрастающая последовательность натуральных чисел $n_1 < n_2 < n_3 < \dots$ Последовательность $b_k = a_{n_k}$ называется \textbf{подпоследовательностью} последовательности $\{a_n\}^{\infty}_{n=1}$.

    Число $a \in \RR$ называется \textbf{частичным пределом} последовательности $\{a_n\}^{\infty}_{n=1}$, если выполнено $a =  \lim_{k \to \infty} a_{n_k}$ для некоторой подпоследовательности $\{a_{n_k}\}^{\infty}_{k=1}$.
    \subsection{Что такое верхний и нижний пределы последовательности? [3.21]}
    Рассмотрим последовательности $M_n :=  \sup_{k > n} a_k$ и $m_n :=  \inf_{k > n} a_k$. Ясно, что последовательность $M_n$ --- не возрастает, а последовательность $m_n$ --- не убывает. Поэтому для ограниченной последовательности $\{a_n\}^{\infty}_{n=1}$ существуют пределы
    \begin{equation*}
        \overline{\lim_{n \to \infty}} a_n := \lim_{n \to \infty} M_n, \underset{n \to \infty}{\underline{\lim}} a_n := \lim_{n \to \infty} m_n,
    \end{equation*}
    которые называются соотвественно \textbf{верхним и нижним частичными пределеами} последовательности $\{a_n\}^{\infty}_{n=1}$.
    \subsection{Приведите формулировку теоремы о том, какие значения может принимать частичный предел ограниченной последовательности. [3.22]}
    Пусть $\{a_n\}^{\infty}_{n = 1}$ -- ограниченная последовательность, тогда $\underset{n \to \infty}{\overline{\lim}} a_n$ и $\underset{n \to \infty}{\underline{\lim}} a_n$ -- частичные пределы последовательности $\{a_n\}^{\infty}_{n=1}$ и любой другой частичный предел принадлежит отрезку $\left[\underset{n \to \infty}{\underline{\lim}\ a_n}, \underset{n \to \infty}{\overline{\lim}} a_n\right]$
    \subsection{Сформулируйте теорему Больцано [3.23]}
    Во всякой ограниченной последовательности можно найти сходящуюся последовательность. 
    \subsection{Сформулируйте критерий сходимости последвательности в терминах частичных пределов. [3.24]}
    Ограниченная последовательность сходится тогда и только тогда, когда множество ее частичных пределов состоит из одного элемента.
    \subsection{Что такое фундаментальная последовательность? [3.25]}
    Говорят, что последовательность $\{a_n\}^{\infty}_{n=1}$ \textbf{фундаментальна} (или является последовательностью Коши), если для каждого числа $\eps > 0$ найдется такое натуральное число (номер) $N(\eps) \in \NN$, что $|a_n - a_m| \eps$ при каждых $n, m > N(\eps)$. То же самое утрвеждение можно переписать в кванторах: 
    \begin{equation*}
        \forall \eps > 0 \ \exists N(\eps) \in \NN : \forall n, m > N(\eps) |a_n - a_m| < \eps
    \end{equation*}

    \textbf{Пример}

    1) Последовательность $a_n = \frac{1}{n}$ -- фундаментальная
    
    2) Последовательность $a_n = (-1)^n$ не фундаментальная
    \subsection{Сформулируйте критерий Коши для последовательности. [3.28]}
    Числовая последовательность сходится тогда и только тогда, когда она фундаментальна.
    \subsection{Что такое числовой ряд? [3.31]}
    Пусть $\{a_n\}^{\infty}_{n=1}$ --- числовая последовательность. \textbf{Числовым рядом} с членами $a_n$ называется выражение 
    \begin{equation*}
        a_1 + a_2 + a_3 + \dots = \sum_{k=1}^{\infty} a_k
    \end{equation*}
    Конечные суммы $S_n := \sum_{k=1}^{n} a_k$ называют \textbf{частичными суммами} ряда $\sum_{k=1}^{\infty} a_k$.

    Говорят, что ряд $\sum_{k=1}^{\infty} a_k$ \textbf{сходится}, если у последовательности $\{S_n\}^{\infty}_{n=1}$ существует предел, который называют суммой ряда. Если такого предела не существует, то говорят, что ряд не сходится или \textbf{расходится}.

    В силу арифметики предела на сходимость ряда (но не на сумму ряда) не влияет добавление (или отбрасывание) первых нескольких слагаемых.

    Также часто бывает удобно индексировать суммирование не только натуральным рядом. Под выражением $\sum_{k=k_0}^{\infty} a_k$ естественным образом подразумевается $\sum_{k=1}^{\infty} a_{k_0 + k-1}$ т.е. такой числовой ряд, чья $n$-я частичная сумма имеет вид $S_n = a_{k_0} + a_{k_0 + 1} + \dots + a_{k_0 + n -1}$
    \subsection{Сформулируйте критерий Коши для числовых рядов. [3.33]}
    Ряд $\sum_{k=1}^{\infty}$ сходится тогда и только тогда, когла для каждого $\eps > 0$ найдется такой номер $N$, что для всех $n > m > N$ выполнено $\left|\sum_{k = m + 1}^{n} a_k\right| = |S_n - S_m| < \eps$
    \subsection{Каково необходимое условие сходимости числового ряда? [3.34]}
    Если ряд $\sum_{k=1}^{\infty} a_k$ сходится, то $a_k \to 0$ при $k \to \infty$.
    \subsection{Что такое гармонический ряд? Сходится ли он?}
    Гармонический ряд --- $\sum_{k=1}^{\infty} \frac{1}{k}$. Он расходится.
    \subsection{Определите абсолютную и условную сходимость ряда. [3.37]}
    Говорят, что ряд $\sum_{k=1}^{\infty} a_k$ \textbf{сходится абсолютно}, если сходится ряд $\sum_{k=1}^{\infty} |a_k|$.

    Говорят, что ряд $\sum_{k=1}^{\infty} a_k$ \textbf{сходится условно}, если он сходится, а ряд $\sum_{k=1}^{\infty} |a_k|$ расходится.
    \subsection{Приведите формулировку теоремы об ограниченности частичных сумм сходящегося ряда с неотрицательными членами. [3.39]}
    Пусть $a_k \geq 0$, тогда ряд $\sum_{k=1}^{\infty} a_k$ сходится тогда и только тогда, когда последовательность его частичных сумм ограничена.
    \subsection{Сформулируйте признак сравнения для числовых рядов. [3.40]}
    Пусть $0 \leq a_n \leq b_n$. Если ряд $\sum_{k=1}^{\infty} b_k$ сходится, то сходится и ряд $\sum_{k=1}^{\infty} a_k$. Наоборот, если ряд $\sum_{k=1}^{\infty} a_k$ расходится, то расходится и ряд $\sum_{k=1}^{\infty} b_k$. 
    \subsection{Сформурийте признак Коши (Лобачевеского-Коши). [3.41]}
    Пусть $\{a_n\}^{\infty}_{n=1}$ -- невозрастающая последовательность $a_n \geq 0$. Ряд $\sum_{k=1}^{\infty} a_k$ сходится тогда и только тогда, когда сходится ряд $\sum_{k=1}^{\infty} 2^k a_{2^k}$
    \subsection{При каких знвчениях параметра $p$ сходится и расходится ряд $\sum_{n=1}^{\infty} \frac{1}{n^p}?$ ($p$-ряды).[3.42]}
    Ряды $\sum_{k=1}^{\infty} \frac{1}{k^p}$ сходится при $p > 1$ и расходится при $p \leq 1$
    \subsection{Сформулируйте определение перестановки слагаемых в рядах. [3.43]}
    Будем говорить, что ряд $\sum_{j=1}^{\infty} \widetilde{a}_{j}$ получен перестановкой членов ряда $\sum_{k=1}^{\infty} a_k$, если существует последовательность натуральных чисел $\{k_j\}^{\infty}_{j=1}$, задающая взаимно однозначное преобразование множества $\NN$, и такая, что $\forall j \in \NN \ \widetilde{a}_j = a_{k_j}$
    \subsection{Сформулируйте теорему Римана. [3.45]}
    Если ряд $\sum_{k=1}^{\infty} a_k$ сходится условно, то для любого $x \in \RR \cup \{+\infty, -\infty\}$ можно так переставить члены ряда $\sum_{k=1}^{\infty} a_k$, что полученный ряд $\sum_{j=1}^{\infty} \widetilde{a}_j$ будет иметь сумму, равную $x$.
    \subsection{Какие множества называют открытыми?}
    Множество открыто, если любая его точка является внутренней.

    Т.е. пусть $U \subset \RR^n$. Тогда $U$ называется открытым, если $\forall x_0 \in U \exists \eps > 0$, такое, что $V_{\eps} (x_0) \subset U$, где $V_{\eps} (x_0) \equiv \{x \in \RR^n \mid  ||x - x_0 || < \eps\}$ - $\eps$-окрестность точки $x_0$
    \subsection{Какие множества называют замкнутыми?}
    Замкнутое множество - множество, которое содержит все свои предельные точки (или множество $A$ замкнуто, если его дополнение $\RR \setminus A$ открыто).
    \subsection{Сформулируйте четыре эквивалентных описания замкнутого множества.}
    блин я не знаю(((((
    \subsection{Что означает термин "внутренняя точка множества"? [4.2]}
    Точка $a \in \RR$ называется \textbf{внутренней} точкой множества $M$, если она входит в это множество $M$ с некоторой своей окрестностью (т.е. надйется такое $\eps > 0$, что $B_{\eps}(a) \subset M$).
    \subsection{Что означает термин "предельная точка множества"? [4.2]}
     Точка $a \in \RR$ называется \textbf{предельной} точкой множества $M$, если каждая ее проколотая окрестность имеет непустое пересечение с множеством $M$ (т.е. для каждого $\eps > 0$ пересечение $B_{\eps}'(a) \cap M \ne 0$)
            \subsection{Что означает термин "граничная точка множества"? [4.2]}
Точка $a \in \RR$ называется \textbf{граничной} точкой множества $M$, если каждая ее окрестность имеет непустое пересечение как с множеством $M$, так и с его дополнением (т.е. для каждого $\eps > 0$ и $B_{\eps}(a) \cap M \ne 0$ и $B_{\eps}(a) \cap (\RR \setminus M) \ne 0$)
    \subsection{Приведите определения предела функции(по множеству) по Коши. [4.6]}
    \subsubsection{Предел функции по Коши}
    Пусть функция $f$ определена в некоторой проколотой окресности $B'(a)$ точки $a$. Число $A$ называется пределом функции $f$ в точке $a$, если для каждого $\eps > 0$ найдется такое $\delta > 0$, что $|f(x)  - A| < \eps$ для каждого $x \in B_{\delta}'(a)$. Используют обозначения $\lim_{x \to a} f(x) = A$ или $f(x) \to A$ при $x \to a$
    \subsubsection{Предел функции по множеству (по Коши)}
    Пусть функция $f$ на некотором множестве $D \subset \RR$ и пусть $a$ предельная для $D$ точка. Число $A$ называется пределом функции $f$ в точке $a$ (по множеству $D$), если для каждого $\eps > 0$ найдется такое $\delta > 0$, что $|f(x) - A| < \eps$ для каждого $x \in D \cap B_{\delta}'(a)$. Используют обозначения $\lim_{x \to a} f(x) = A$ или $f(x) \to A$ при $x \to a$.

    т.е. $\lim_{x \to a} f(x)= A$, если 
    \begin{equation*}
        \forall \eps > 0 \exists \delta > 0 : \forall x \in D \cap B_{\delta}'(a) \ \ |f(x) - A| < \eps
    \end{equation*}
    или
    \begin{equation*}
        \forall \eps > 0 \exists \delta > 0 : \forall x \in D, 0 < |x - a| < \delta \ \ |f(x) - A| < \eps
    \end{equation*}
    \subsection{Приведите определния предела функции (по множству) по Гейне. [4.5]}
    \subsubsection{Предел функции по Гейне.}
    Пусть функция $f$ определена в некоторой проколотой окрестности $B'(a)$ точки $a$. Число $A$ называется пределом функции $f$ в точке $a$, если для каждой последовательности точек $x_n \in B'(a), x_n \to a$, выполнено $f(x_n) \to A$ при $n \to \infty$
    \subsubsection{Предел функции по множеству(по Гейне)}
    Пусть функция $f$ определена на некотором множестве $D \subset \RR$ и пусть $a$ предельная для $D$ точка. Число $A$ называется пределом фнукции $f$ в точке $a$ по множеству $D$, если для каждой последовательности точек $x_n \in D \setminus \{a\}, x_n \to a,$ выполнено $f(x_n) \to A$ при $n \to \infty$.
    \subsection{Сформулируйте теорему о пределе сложной функции. [4.13]}
    Пусть $f \colon D \to E, g \colon E \to \RR, a$ -- предельная точка множества $D, b$ -- предельная точка множества $E, \lim_{x \to a} f(x) = b, \lim_{y \to b} g(y) = c$ и есть такая проколотая окрестность $B_{\delta}' (a)$ точки $a$, что $f(x) \ne b$ для каждой точки $x \in B_{\delta}'(a) \cap D$. Тогда $\lim_{x \to a} g(f(x)) = c$
    \subsection{Как выглядит первый замечательный предел? [4.14]}
    $\lim_{x \to 0} \frac{\sin x}{x} = 1$
    \subsection{Как выглядит второй замечательный предел? [4.15]}
    $\lim_{x \to + \infty} \left(1 + \frac{1}{x}\right)^x = e$
    \subsection{Сформируйте критерий Коши существования предела функции. [4.16]}
    Пусть $f \colon D \to \RR$ и $a$ - предельная точка $D$. Предел $\lim_{x \to a} f(x)$ существует тогда и только тогда когда для каждого $\eps > 0$ найдется такое $\delta > 0$, что для каждых $x, y \in B_{\delta}'(a) \cap D$ выполнено $|f(x) - f(y)| < \eps$
    \subsection{Определите понятие одностороненного предела функции. [4.17]}
    Пусть здесь и далее $D^{+}_a := D \cap (a, + \infty)$ и $D^{-}_a := D \cap (-\infty, a)$

    Пусть точка $a$ -- предельная для множества $D^{+}_a$ и существует предел функции $f$ по множеству $D^{+}_a$ в точке $a$. Этот предел называют пределом справа фнукции $f$ в точке $a$ и обозначают $\lim_{x \to a + 0} f(x)$ или просто $f(a + 0)$. Аналогично определяется предел слева, который обозначают $\lim_{x \ to a - 0} f(x)$ или $f(a - 0)$
    \subsection{Сформулируйте теорему Вейерштрасса о существовании односторонних пределов монотонной ограниченной функции. [4.18]}
    Пусть $f$ -- не убывает и ограничена на множестве $D, a$ -- предельная точка $D^{-}_a$. Тогде существует предел слева
    \begin{equation*}
        \lim_{x \to a - 0} f(x) = \sup \{f(x) \colon x \in D^{-}_a\}
    \end{equation*}
    Пусть $f$ - не убывает и ограничена на множестве $D, a$ -- предельная точка множества $D^{+}_a$. Тогда существует предел справа
    \begin{equation*}
        \lim_{x \to a + 0} f(x) = \inf \{f(x) \colon x \in D^{+}_a\}.
    \end{equation*}
    Аналогичные утверждения с заменой $\inf$ на $\sup$ справедлины и для невозрастающей функции.
    \subsection{Определите отношение эквивалентности функций.[4.24]}
    Пусть функции $f$ и $g$ определены и не обращаются в 0 в некоторой $B'(x_0).$ Функции $f$ и $g$ называются эквивалентными (пишут: $f(x) \sim g(x)$) \ \ при $x \to x_0$, если \ \ $\lim_{x \to x_0} \frac{f(x)}{g(x)} = 1.$
    \subsection{Что означает, что функция является бесконечно малой относительно другой ($\bar{o}$)?[4.31]}
    Пусть функции $f$ и $g$ определены в $B'(x_0)$ и функция $g(x)$ не обращается в 0. Говорят, что функция $f$ является бесконечно малой относительно функции $g$ при $x \to x_0$ и пишут $f(x) = o(g(x))$ при $x \to x_0$, если $\lim_{x \to x_0} \frac{f(x)}{g(x)}=0$
    \subsection{Что означает, что функция является ограниченной относительно другой $(\underline{O})$?[4.33]}
    Пусть функции $f$ и $g$ определены в $B_{\delta}' (x_0)$. Говорят, что функция $f$ ограничена относительно функции $g$, и пишут $f(x)$ = $O(g(x))$ при $x \to x_0$, если 
    \begin{equation*}
        \exists C \in \RR : \forall x \in B_{\delta}' (x_0) \ \ |f(x)| \leq C|g(x)|
    \end{equation*}
    \subsection{Сформулируйте теорему о связи отношения эквивалентности и представлеия функции с помощью $\bar{o}$ [4.32]}
    $f(x) \sim g(x)$ при $x \to x_0 \Leftrightarrow f(x) - g(x) = o(g(x))$ при $x \to x_0$.
    \subsection{Приведите любые три свойства $\bar{o}$ и $\underline{O}$ из семинарской задачи или теоремы [4.34.]}
    \begin{enumerate}
        \item $o(f) \pm o(f) = o(f)$
        \item $O(f) \pm O(f) = O(f)$
        \item $o(f) = O(f)$
        \item $o(O(f)) = o(f)$
        \item $O(o(f)) = o(f)$
        \item $O(O(f)) = O(f)$
        \item $o(f) \cdot O(g) = o(fg)$
        \item $O(f) \cdot O(g) = O(fg)$
        \item $(o(f))^{\alpha} = o(f^{\alpha}) \ \forall \alpha > 0$
        \item $(O(f))^{\alpha} = O(f^{\alpha}) \ \forall \alpha > 0$
    \end{enumerate}
    \section{Теоретические вопросы на доказательство}
    \subsection{Представление вещественных чисел в виде десятичных дробей [2.4], [2.6]. Принцип полноты, его выполнение для десятичных дробей [2.7]}
    \subsubsection{Представление вещественных чисел в виде десятичных дробей.}
    Вещественное число определяется как бесконечная десятичная дробь, то есть выражение вида $\pm a_0, a_1 a_2 \dots a_n \dots$, где $\pm$ есть один из символов + или -, называемый знаком числа, $a_0$ - целое неотрицательное число, $a_1, a_2, \dots a_n, \dots$ -- последовательность десятичных знаков, то есть элементов числового множества $\{0, 1, \dots, 9\}$
    
    \textbf{не очень понимаю, что здесь доказывать}
    \subsubsection{Сравнение бесконечных десятичных дробей}
    Сравнение вещественных чисел в форме бесконечных десятичных дробей производится поразрядно. Например, пусть даны два неотрицательных числа 
    \begin{equation*}
        \alpha = + a_0, a_1, a_2 \dots a_n \dots
    \end{equation*}
    \begin{equation*}
        \beta = +b_0, b_1, b_2 \dots b_n \dots
    \end{equation*}
    Если $a_0 < b_0$, то $\alpha < \beta$;, если $a_0 > b_0$, то $\alpha > \beta$. В случае равенства $a_0 = b_0$ переходит к сравнению следующего разряда. И так далее. Если $\alpha \ne \beta$, то после конечного числа шагов встретится первый разряд $n$, такой, что $a_n \ne b_n$. Если $a_n < b_n$, то $\alpha < \beta$; если $a_n > b_n$, то $\alpha > \beta$

    \textbf{аналогично не понимаю, что тут доказывать..}
    \subsubsection{Принцип полноты, его выполнение для десятичных дробей}
    При таком определении для вещественных чисел выполнена аксиома непрерывности.
    
    \textbf{Теорема 2.7} Если $A \leq B (\forall a \in A \ \forall b \in B \ \ a \leq b)$
    
    То $\exists c: \ \ \forall a \in A \ \forall b \in B \ \ a \leq c \leq b$
    \begin{proof}

        1) $\forall a \subset A \ \ a \leq 0$ и $\forall b \in B \ b \geq 0$

        Тогда $c = 0$

        2) $\begin{array}{c}
            2.1\\
            2.2
        \end{array}\left[\begin{array}{c}
            \exists a \in A \ \ a > 0 (\to \forall b \in B \ b > 0)\\
            \exists b \in B \ \ b < 0 (\to \forall a \in A \ a < 0)
        \end{array}\right.$

        $A' = -B (=\{x \mid x = -b \ b \in B\})$

        $B' = -A$
        
        $2.1 \to 2.2$

        2.1: $c = c_0, c_1 c_2 \dots c_k \dots$

        $c_0 = \min \{b_0 \mid \exists b \in B \ b = b_0, b_1 b_2 \dots\}$

        $c_1 = \min \{b_1 \mid \exists b \in B \ b = c_0, b_1 b_2 \dots\}$

        $c_2 = \min \{b_2 \mid \exists b \in B \ b = c_0, c_1 b_2 \dots\}$

        $c_k = \min \{b_0 \mid \exists b \in B \ b = c_0, c_1 \dots c_{k-1} b_{k} b_{k+1}\dots\}$

        0) $c \in \RR$ иначе $c_k = c_0, c_1 \dots (9) \to \exists b \in B \ \ b = c_0, c_1 999 \dots (9)$

        1) $\forall b \in B \ c \leq b$ 

        Пусть $\exists b \in B \ \ b < c$

        
    {
        \setlength{\arraycolsep}{1pt}
        $\begin{array}{ccccccc}
        b_0 & b_1 & b_2 & b_3 & \cdots & b_{k-1} & b_k \\[-0.3em]
        \verteq & \verteq & \verteq  & \verteq & & \rotatebox{90}{>} & \\[-0.3em]
        c_0 & c_1 & c_2 & c_3 & \cdots & c_{k-1} & c_k
        \end{array}
        $
    }

    2) $\forall a \in A \ \ c \geq a$

    Пусть $\exists a \in A \ \ a > c$

    {
        \setlength{\arraycolsep}{1pt}
        $\begin{array}{ccccccccc}
            a & = & a_0, & a_1 & a_2 & \dots & a_{k-1} & a_k & \dots\\
            & & \verteq & \verteq & \verteq & \dots & \verteq &\rotatebox{90}{<} & \dots\\
            c & = & c_0, & c_1 & c_2 & \dots & c_{k-1} & c_k & \dots \\
            & & \verteq & \verteq & \verteq & \dots & \verteq & \verteq & \dots\\
            b & = & b_0, & b_1 & b_2 & \dots & b_{k-1} & b_{k} & \dots
        \end{array} \to a > b \ \text{--- противоречие.}$
    }
    \end{proof}
    \subsection{Верхняя и нижняя грань, максимум и минимум множества. [2.9], [2.10] Точные верхние и нижние грани, их существование у ограниченных множеств. [2.12], [2.14], теорема [2.15]}
    \subsubsection{Верхняя и нижняя грань.}
    \begin{enumerate}
        \item Число $M \in \RR$ называется верхней гранью множества $A \subset \RR$, Если число $M$ лежит справа от множества $A$, т.е. $\forall a \in A \ \ A \leq M$.
        \item Множество $A \subset \RR$ называется ограниченным сверху, если существует (конечная) верхняя грань этого множества: $\exists M \in \RR : \forall a \in A \ \ A \leq M$.
        \item Число $m \in \RR$ называется нижней гранью множества $A \subset \RR$, если число $m$ лежит слева от множества, т.е. $\forall a \in A \ \ a \geq m$.
        \item Множество $A \subset \RR$ называется ограниченным снизу, если существует (конечная) нижняя грань этого множества: $\exists m \in \RR : \forall a \in A \ \ a \geq m$.
        \item Множество $A$ называется ограниченным, если $A$ огранчиено сверху и ограничено снизу.
    \end{enumerate}
    \subsubsection{Максимум и минимум множества.}
    \begin{itemize}
        \item $M \in A$,
        \item $M$ является верхней гранью $A$.
    \end{itemize}
    Число $m$ называется минимальным элементом множества $A \subset \RR (m = \min A)$, если
    \begin{itemize}
        \item $m \in A$,
        \item $m$ является нижней гранью $A$.
    \end{itemize}
    \subsubsection{Точные верхние и нижние грани}
    \begin{enumerate}
        \item $M$ является верхней гранью множества $A$ и
        \item не существует числа, меньшего, чем $M$, и являющегося верхней гранью множества $A$, то есть
    \end{enumerate}
    \begin{itemize}
        \item  $\forall a \in A \ \ a\leq M$ и
        \item $\lnot (\exists M' \in \RR : M' < M \ \text{и} \ \forall a \in A \ \ a \leq M')$.
    \end{itemize}

    Число $m \in \RR$ называется точной нижней гранью или инфинумом множества $A \subset \RR (m = \inf A)$, если $m$ является максимальной нижней гранью $A$.
    \subsubsection{существование точных верхних и нижних граней у ограниченных множеств}
    \textbf{Теорема 2.15}

    Пусть множество $A \subset \RR$ ограничено сверху. Тогда существует единственное число $M \in RR$, которое является точнйо верхней гранью множества $A$.
    \begin{proof}
    Рассмотрим $B$ - множество всех (конечных) верхней граней $A$. Так как множество $A$ ограничено сверху, то $B$ не пусто. Поскольку множество $A$ лежит слева от множества $B$, то по аксиоме непрерывности $\exists c \in \RR : \forall a \in A \ \ \forall b \in B \ \ a \leq c \leq b$. Покажем, что $c$ является точной верхней гранью $A$. Так как $\forall a \in A \ \ a \leq c$, то $c$ является верхней гранью $A$, т.е. $c \in B$. Поскольку $\forall b \in B \ \ c \leq b$, то $c$ -- минимальный элемент $B$. Итак, $c$ -- точная верхняя грань $A$.

    Предположим, что $M_1, M_2 \in \RR$ -- две различные точные верхние грани множества $A$. Тогда $M_1, M_2$ -- два различных минимальных элемента множества $B$. Пусть для определенности $M_1 < M_2$. Тогда $M_2$ не является минимальным элементом множества $B$. Противоречие.
    \end{proof}
    \subsection{Предел последовательности, его основные свойства: единственность, ограниченность сходящейся последовательности. [3.1], [3,3], [3.8], [3.7], [3.9]}
    \subsubsection{Предел последовательности}
    Число $a \in \RR$ называется пределом последовательности $\{a_n\}$ (пишут $a =  \lim_{n \to \infty} a_n$ или $a_n \to a$ при $n \to \infty$), если
    \begin{equation*}
        \forall \eps > 0 \ \ \exists N(\eps) : \ \ \forall n \geq N(\eps) \ \ a_n \in B_{\eps}(a), 
    \end{equation*}
    т.е.
    \begin{equation*}
        \forall \eps > 0 \ \ \exists N: \forall n \geq N |a_n - a| < \eps
    \end{equation*}
    Заметим, что в формулах для каждого $\eps > 0$ существует свое число $N$, то есть $N$ зависит от $\eps$. Чтобы подчеркнуть эту зависимость иногда вторую формулу переписывают в виде: 
    \begin{equation*}
        \forall \eps > 0 \ \ \exists N = N(\eps): \ \ \forall n \geq N \ \ |a_n - a| < \eps
    \end{equation*}
    \subsubsection{Единственность предела}
    Пусть $\lim_{n \to \infty} a_n = a$ и $\lim_{n \to \infty} a_n = b$, тогда $a = b$
    \begin{proof}
    Действительно, если $a \ne b$, то $|a - b| = \eps_0 > 0$. Но по определению найдется номер $N_1$, для которого $|a_n - a| < \frac{\eps_0}{2}$ при $n > N_1$ и найдется номер $N_2$, для которого $|a_n - b| < \frac{\eps_0}{2}$ при $n > N_2$. Тогда при $n > \max \{N_1, N_2\}$ выполнено
    \begin{equation*}
        \eps_0 = |a - b| = |a - a_n + a_n - b| \leq |a - a_n| + |a_n - b| < \eps_0.
    \end{equation*}
    Противоречие.
    \end{proof}
    \subsubsection{Ограниченность сходящейся последовательности.}
    Сходящаяся последовательность ограничена.
    \begin{proof}
    Если $\lim_{n \to \infty} a_n = a$, то для некоторого $N \in \NN$ выполнено $|a_n - a| < 1$ при $n > N$. Отсюда $|a_n| = |a_n - a + a| \leq |a_n - a| + |a| < 1 + |a|$ при $n > N$. Значит,
    \begin{equation*}
        |a_n| \leq M = max \{1 + |a|, |a_1|, \dots , |a_{N}|\}
    \end{equation*}
    т.е. $-M = c \leq a_n \leq C = M$
    \end{proof}
    \subsection{Лемма об отделимости. [3.10], арифметические свойства предела последовательности [3.11]}
    \subsubsection{Лемма об отделимости}
    Если $a_n \to a$ и $a \ne 0$, то найдется номер $N \in \NN$, для которого $|a_n| > \frac{|a|}{2} > 0$ при $n > N$

    \begin{proof} Действительно, взяв $\eps = \frac{|a|}{2}$ в определении сходимости последовательности к числу $a$, получаем номер $N \in \NN$, для которого $|a_n - a| < \frac{|a|}{2}$ при $n > N$. Тогда, при $n > N$, выполнено $|a| - |a_n| \leq |a_n - a| < \frac{|a|}{2}$, что равносильно доказываемому утверждению. \end{proof}
    \subsubsection{Арифметические свойства предела последовательности}
     Пусть $ \lim_{n \to \infty} a_n = a$ и $ \lim_{n \to \infty} b_n = b$. Тогда
    \begin{enumerate}
        \item $ \lim_{n \to \infty} (\alpha a_n + \beta b_n) = \alpha a + \beta b \ \ \forall \alpha, \beta \in \RR$;
        \item $ \lim_{n \to \infty} a_n b_n = ab$;
        \item если $b \ne 0, b_n \ne 0$, то $ \lim_{n \to \infty} \frac{a_n}{b_n} = \frac{a}{b}$.
    \end{enumerate}
    \begin{proof}
        Пусть $\eps > 0$ -- произвольное число. Тогда найдется номер $N_1$, для которого $a_n - a < \eps$, и найдется номер $N_2$, для которого $b_n - b < \eps$.

        1) Получаем, что при $n > N = \max \{N_1, N_2\}$ выполнено
        \begin{equation*}
            |\alpha a_n + \beta b_n - (\alpha a + \beta b)| = |\alpha (a_n - a) + \beta (b_n - b)| \leq |\alpha| |a_n - a| + |\beta| |b_n - b| < (|\alpha| + |\beta|)\eps.
        \end{equation*}

        2) Замечаем, что $|a_n b_n - ab| = |a_n b_n - a b_n + a b_n - ab| \leq |b_n||a_n - a| + |a| |b_n - b|.$ Т.к. сходящаяся последовательость ограничена, то найдется число $M > 0$, для которого $b_n \leq M$, поэтому при $n > N = \max \{N_1, N_2\}$ выполнено $|a_n b_n - ab| \leq (M + |a|) \eps$.

    3) Достаточно проверить что $\frac{1}{b_n} \to \frac{1}{b}$ при $n \to \infty$. Заметим, что по условию $b \ne 0$, поэтому найдется номер $N_3 \in \NN$, для которого, при $n > N_3$, выполнено $b_n > \frac{|b|}{2}$. Тогда при $n > \max \{N_2, N_3\}$ выполнено 
    \begin{equation*}
        \left|\frac{1}{b_n} - \frac{1}{b}\right| = \frac{|b_n - b|}{|b_n||b|} \leq \frac{2}{|b|^2} \cdot \eps
    \end{equation*}
    \end{proof}
    \subsection{Переход к пределу в неравенствах и лемма о зажатой последовательности. [3.12], [3.13]}
    \subsubsection{О переходе к пределу в неравенстве}
    Пусть $ \lim_{n \to \infty} a_n = a,  \lim_{n \to \infty} b_n = b$. Если для некоторого номера $N$ выполнено $a_n \leq b_n$ при $n > N$, то $a \leq b$.
    \begin{proof}
        Предположим, что $a - b = \eps_0 > 0$. Тогда найдутся номера $N_1 \in \NN$ и $N_2 \in \NN$, для которых $|a_n - a| < \frac{\eps_0}{2}$ при $n > N_1$ и $|b_n - b| < \frac{\eps_0}{2}$ при $n > N_2$. Тогда 
        \begin{equation*}
            \eps_0 = a - b = a - a_n + a_n - b_n + b_n - b \leq a - a_n + b_n - b \leq \eps_0
        \end{equation*}
    \end{proof}
    \subsubsection{Лемма о зажатой последовательности (двух миллиционерах)}
    Пусть $ \lim_{n \to \infty} a_n =  \lim_{n \to \infty} b_n = a$ и для некоторого $N \in \NN$ выполнено $a_n \leq c_n \leq b_n$ при $n > N$. Тогда $ \lim_{n \to \infty} c_n = a$
    \begin{proof}
    Для каждого $\eps > 0$ найдутся номера $N_1 \in \NN$ и $N_2 \in \NN$, для которых $|a_n - a| < \eps$ и $|b_n - a| < \eps$. Тогда при $n > \max \{N, N_1, N_2\}$ выполнено
    \begin{equation*}
        a - \eps < a_n \leq c_n \leq b_n < a + \eps
    \end{equation*}
    \end{proof}
    \subsection{Теорема Вейерштрасса о пределе монотонной ограниченной последовательности [3.14]}
    Пусть последовательность $\{a_n\}^{\infty}_{n=1}$ не убывает $(a_n \leq a_{n + 1})$ и ограничена сверху. Тогда эта последовательность сходится к своему супремуму.

    Аналогично, пусть последовательность $\{a_n\}^{\infty}_{n=1}$ не возрастает $(a_{n+1} \leq a_n)$ и ограничена снизу. Тогда эта последовательность сходится к своему инфимуму.
    \begin{proof}
    Докажем только первое утверждение, второе доказывается аналогично или переходом к последовательности $\{-a_n\}^{\infty}_{n=1}$.

    Пусть $M = \sup \{a_n \colon n \in \NN \} = \sup a_n$. Тогда для каждого $\eps > 0$ найдется номер $N \in \NN$, для которого $M - \eps < a_{N}$ (иначе $M - \eps$ --- верхняя грань, чего не может быть). В силу того, что последовательность неубывающая, при каждом $n > N$ выполнено
    \begin{equation*}
        M - \eps < a_{N} \leq a_n \leq M < M + \eps
    \end{equation*}
    Тем самым, по определени $M = \lim a_n$
    \end{proof}
    \subsection{Вычисление $\sqrt{2}$ с помощью рекуррентной формулы $a_{n+1} = \frac{1}{2} \left(a_n + \frac{2}{a_n}\right)$, обоснование сходимости [3.15]}
    Пусть $a_{n + 1} = \frac{1}{2} \left(a_n + \frac{2}{a_n}\right), a_1 = 2$. Тогда последовательность $\{a_n\}^{\infty}_{n=1}$ сходится и ее предел равен $\sqrt{2}$
    \begin{proof}
    Заметим, что 
    \begin{equation*}
        a_{n + 1} = \frac{1}{2} \left(a_n + \frac{2}{a_n}\right) \geq \frac{1}{2} \cdot 2\sqrt{a_n \cdot \frac{2}{a_n}} = \sqrt{2}
    \end{equation*}
    Поэтому $a_n \geq \sqrt{2}$. Кроме того $a_{n + 1} = \sqrt{1}{2} \left(a_n + \sqrt{2}{a_n}\right) \leq \frac{1}{2} \left(a_n + \frac{a_n^2}{a_n}\right) = a_n$. По доказанной теореме последовательности $\{a_n\}^{\infty}_{n=1}$ существует предел $a$. Т.к. $a_n \geq \sqrt{2} > 0$, то и $a > 0$. Тогда, по арифметике предела получаем $a = \frac{1}{2} \left(a + \frac{2}{a}\right)$, откуда $a = \sqrt{2}$. Исследуем теперь скорость сходимости:
    \begin{equation*}
        \left|a_{n + 1} - \sqrt{2}\right| = \frac{|a^2_n - 2a_n \sqrt{2} + 2|}{2 a_n} = \frac{(a_n - \sqrt{2})^2}{2a_n} \leq \frac{(a_n - \sqrt{2})^2}{2 \sqrt{2}} \leq \left(a_n - \sqrt{2}\right)^2
    \end{equation*}
    Индуктивно Получаем
    \begin{equation*}
        \left|a_{n + 1} - \sqrt{2}\right| \leq \left(a_n - \sqrt{2}\right)^2 \leq \left(a_{n-1} - \sqrt{2}\right)^4 \leq \left(a_{n-2} - \sqrt{2}\right)^8 \leq \left(a_1 - \sqrt{2}\right)^{2^{n + 1}} = (2 - \sqrt{2})^{2^{n+1}}.
    \end{equation*}
    Заметим, что $q := 2 - \sqrt{2} < 1$, поэтому полученная скорость сходимость $q^{2n}$ быстрее экспоненциальной $q^n$ (в смысле количества применений рекуррентной формулы для достижения заднной точки.)
    \end{proof}
    \subsection{Число $e$ --- определение и обоснование корректности. [3.16], [3.17]}
    Предел последовательности $a_n = \left(1 + \frac{1}{n}\right)^n$ называют \textbf{числом Эйлера} и обозначают
    \begin{equation*}
         \lim_{n \to \infty} \left(1 + \frac{1}{n}\right)^n = e
    \end{equation*}
    \begin{proof}
    Пусть $a_n = (1 + \frac{1}{n})^n$. По биному Ньютона
    \begin{equation*}
        a_n = \sum_{k=0}^{n} C^k_n \frac{1}{n^k} = 2 + \sum_{k=2}^{n} \frac{1}{k!} \frac{n \cdot (n - 1) \cdot \dots \cdot (n - k + 1)}{n^k} = 2 + \sum_{k=2}^{n} \frac{1}{k!} \left(1 - \frac{1}{n}\right) \cdot \dots \cdot \left(1 - \frac{k-1}{n}\right).
    \end{equation*}
    Отсюда, во-первых, получаем, что 
    \begin{equation*}
        a_n \leq 2 + \sum_{k=2}^{n} \frac{1}{k!} \leq 2 + \sum_{k=2}^{n} \frac{1}{2^{k-1}} < 3
    \end{equation*}
    где было использовано неравенство $k! = 1 \cdot 2 \cdot 3 \cdot \dots \cdot k \geq 2^{k-1}$ при $k \geq 2$. Во-вторых, 
    \begin{equation*}
        a_n = 2 + \sum_{k=2}^{n} \frac{1}{k!} \left(1 - \frac{1}{n}\right)\cdot \dots \cdot \left(1 - \frac{k - 1}{n}\right) \leq 2 + \sum_{k=2}^{n} \frac{1}{k!} \left(1 - \frac{1}{n + 1}\right)\cdot \dots \cdot \left(1 - \frac{k-1}{n + 1}\right)
    \end{equation*}
    \begin{equation*}
        \leq 2 + \sum_{k=2}^{n + 1} \frac{1}{k!}\left(1 - \frac{1}{n + 1}\right) \cdot \dots \cdot \left(1 - \frac{k-1}{n + 1}\right) = a_{n + 1}.
    \end{equation*}
    таким образом, последовательность $a_n$ -- не убывает и ограничена сверху, а значит имеет предел.
    \end{proof}
    \subsection{Принцип вложенных отрезков [3.18], геометрическая интерпретация вещественных чисел.}
    Всякая последовательность $\{[a_n, b_n]\}^{\infty}_{n=1}$ вложенных отрезков (т.е. $[a_{n + 1}, b_{n + 1}] \subset [a_n, b_n]$) имеет общую точку. Кроме того, если длины отрезков стремятся к нулю, т.е. $b_n - a_n \to 0$, то такая общая точка только одна.
    \begin{proof}
    По условию $[a_{n + 1}, b_{n + 1}] \subset [a_n, b_n]$, откуда $a_n \leq a_{n + 1} \leq b_{n + 1} \leq b_n$. Заметим, что при $n < m$ выполнено $a_n \leq a_m \leq b_m$, а при $n > m$ выполнено $a_n \leq b_n \leq b_m$. Таким образом, если $A := \{a_n, n \in \NN\}$ и $B := \{b_m, m \in \NN\}$, то $A$ левее $B$, а значит, по принципу полноты найдется такое число $c \in \RR$, что $a_n \leq c \leq b_m$ для проивзольных $n, m \in \NN$. В частности $a_n \leq c \leq b_n$, т.е. $c \in [a_n, b_n]$.

    Пусть общих точек две: $c$ и $c'$. Не ограничивая общности, $c < c'$. Тогда $a_n \leq c \leq c' \leq b_n$ и $c' - c \leq b_n - a_n$, что противоречит тому, что $\lim_{n \to \infty} (b_n - a_n) = 0$. Действительно, найдется номер $N \in \NN$, для которого $b_n - a_n < c' - c$ при каждом $n > N$.
    \end{proof}
    \subsection{Подпоследовательность и частичные пределы [3.19], [3.20]. Верхний и нижни йпределы ограниченной последовательности [3.21], их связь с множеством частичных пределов этой последовательности [3.22]}
    \subsubsection{Частичный предел}
     Пусть задана последовательность $\{a_n\}^{\infty}_{n=1}$ и пусть задана возрастающая последовательность натуральных чисел $n_1 < n_2 < n_3 < \dots$ Последовательность $b_k = a_{n_k}$ называется \textbf{подпоследовательностью} последовательности $\{a_n\}^{\infty}_{n=1}$.

    Число $a \in \RR$ называется \textbf{частичным пределом} последовательности $\{a_n\}^{\infty}_{n=1}$, если выполнено $a =  \lim_{k \to \infty} a_{n_k}$ для некоторой подпоследовательности $\{a_{n_k}\}^{\infty}_{k=1}$.
    \subsubsection{Сходимость подпоследовательности.}
    Любая подпоследовательность сходящейся последовательности сходится к пределу этой последвательности.
    \begin{proof}
    Рассмотрим последовательность $\{a_n\}_{n=1}^{\infty}$. Пусть $\lim_{n \to \infty} a_n = a$ и пусть $\{a_{n_k}\}_{k=1}^{\infty}$ --- подпоследовательность. По определению предела для каждого $\eps > 0$ найдется номер $N$, для которого $a_n - a < \eps$ при $n > N$. Т.к. $1 \leq n_1$ и $n_{k-1} < n_k$, по индукции получаем, что $k \leq n_k$. Поэтому $|a_{n_k} - a| < \eps$ при $k > N$.
    \end{proof}
    \subsubsection{Верхний и нижний пределы}
    Рассмотрим последовательности $M_n :=  \sup_{k > n} a_k$ и $m_n :=  \inf_{k > n} a_k$. Ясно, что последовательность $M_n$ --- не возрастает, а последовательность $m_n$ --- не убывает. Поэтому для ограниченной последовательности $\{a_n\}^{\infty}_{n=1}$ существуют пределы
    \begin{equation*}
        \overline{\lim_{n \to \infty}} a_n := \lim_{n \to \infty} M_n, \underset{n \to \infty}{\underline{\lim}} a_n := \lim_{n \to \infty} m_n,
    \end{equation*}
    которые называются соотвественно \textbf{верхним и нижним частичными пределеами} последовательности $\{a_n\}^{\infty}_{n=1}$.
    \subsubsection{Связь верхних и нижних пределов с множеством частичных пределов этой последовательности}
    Пусть $\{a_n\}^{\infty}_{n = 1}$ -- ограниченная последовательность, тогда $\underset{n \to \infty}{\overline{\lim}} a_n$ и $\underset{n \to \infty}{\underline{\lim}} a_n$ -- частичные пределы последовательности $\{a_n\}^{\infty}_{n=1}$ и любой другой частичный предел принадлежит отрезку $\left[\underset{n \to \infty}{\underline{\lim}\ a_n}, \underset{n \to \infty}{\overline{\lim}} a_n\right]$
    \begin{proof}
    Покажем, что $M := \underset{n \to \infty}{\overline{\lim}} a_n$ --- частичный предел, для этого рекуррентно задаим подпоследовательность, которая сходится к $\underset{n \to \infty}{\overline{\lim}} a_n$. Пусть $n_1 = 1$. Пусть индексы $n_1 < n_2 < \dots < n_k$ уже построены. Тогда подберем такой номер $n_{k + 1} > n_k$, что 
    \begin{equation*}
        M_{n_k} - \frac{1}{k} < a_{n_{k+1}} \leq M_{n_k}
    \end{equation*}
    Как подпоследовательность сходящейся последовательности $M_{n_k} \to M$, поэтому по теореме о сходимости зажатой последовательности получаем, что $\lim_{k \to \infty} a_{n_k} = M$. Аналогично проверяется, что $\underset{n \to \infty}{\underline{\lim}} a_n$ --- частичный предел.

    Пусть теперь $a$ --- частичный предел. Это означает что $a = \lim_{k \to \infty} a_{n_k}$ для некоторой подпоследовательности $\{a_{n_k}\}^{\infty}_{k=1}$. Тогда $m_{n_{k-1}} \leq a_{n_k} \leq M_{n_{k-1}}$. По теореме о переходе к пределу в неравенствах получаем, что $\underset{n \to \infty}{\underline{\lim}} a_n \leq a \leq \underset{n \to \infty}{\overline{\lim}} a_n$ 
    \end{proof}
    \subsection{Теорема Больцано [3.23]. Критерий сходимости последовательности в терминах частичных пределов [3.24]}
    \subsubsection{Теорема Больцано-Вейерштрасса}
    Во всякой ограниченной последовательности можно найти сходящуюся последовательность. 
    \begin{proof}
    \textbf{Вообще напрямую следует из предыдущей теоремы, но можно доказать и независимо.}

    Пусть $a_n$ --- ограниченная последовательность. Тогда все члены последовательности лежат на отрезке $[c, C]$. Тогда на этом отрезке бесконечное число членов этой последовательности. Режем отрезок пополам, тогда как минимум один из получившихся отрезков содержит бесконечное количество членов, обозначим его как $[a_1, b_1]$. Проделаем с ним то же самое и обозначим $[a_2, b_2]$. Проделав эту операцию получаем последовательность вложенных отрезков: $[c, C] \supset [a_1, b_1] \supset [a_2, b_2] \supset \dots [a_n, b_n] \supset [a_{n+1}, b_{n+1}]$. Длина этих отрезков стремится к нулю, тогда существует единственная точка $c$, общая для всех этих отрезков. Возьмем как $n_1$ любой член отрезка $[a_1, b_1]$, далее возьмем $n_2$ из $[a_2, b_2]$, проделаем эту операцию еще много раз. Именно последовальельность из этих $n$-ок будет сходится к этой общей точке.
    \end{proof}
    \subsubsection{Критерий сходимости последовательности в терминах частичных пределов}
    Ограниченная последовательность схдится тогда и только тогда, когда множество ее частичных пределов состоит из одного элемента.
    \begin{proof}
    То, что у сходящейся последовательности есть единственный частичный предел уже проверено ранее.

    Предположим, что у ограниченной последовательности $\{a_n\}^{\infty}_{n=1}$ существует единственный частичный предел. По доказанному, это в частности означает, что
    \begin{equation*}
        \underset{n \to \infty}{\overline{\lim}} a_n = \underset{n \to \infty}{\underline{\lim}}a_n = a
    \end{equation*}
    Тогда $m_{n-1} \leq a_n \leq M_{n-1}$. И по теореме о сходимости зажатой последовательности получаем, что $\lim_{n \to \infty} a_n = a$
    \end{proof}
    \subsection{Фундаментальная последовательность и критерий Коши. [3.25], [3.27], [3.29]}
    \subsubsection{Фундаментальная последовательность}
    Говорят, что последовательность $\{a_n\}^{\infty}_{n=1}$ \textbf{фундаментальна} (или является последовательностью Коши), если для каждого числа $\eps > 0$ найдется такое натуральное число (номер) $N(\eps) \in \NN$, что $|a_n - a_m| \eps$ при каждых $n, m > N(\eps)$. То же самое утрвеждение можно переписать в кванторах: 
    \begin{equation*}
        \forall \eps > 0 \ \exists N(\eps) \in \NN : \forall n, m > N(\eps) |a_n - a_m| < \eps
    \end{equation*}
    \subsubsection{Связь фундаментальности и сходимости.}
    Если последовательность $\{a_n\}^{\infty}_{n=1}$ сходится, то она фундаментальна.
    \begin{proof}
    Пусть $\eps > 0$. По определению сходящейся последовательности найдется такой пример $N \in \NN$, что $|a_n - a| < \frac{\eps}{2}$ при $n > N$, где $a := \lim_{n \to \infty} a_n$. Тогда при $n, m > N$ выполнено
    \begin{equation}
        |a_n - a_m| = |a_n - a + a - a_m| \leq |a_n - a| + |a - a_m| < \eps
    \end{equation}
    \end{proof}
    \subsubsection{Критерий Коши}
    Числовая последовательность сходится тогда и только тогда, когда она фундаментальна.
    \begin{proof}
    Во-первых, заметим, что последовательность $\{a_n\}^{\infty}_{n=1}$ ограничена. Действительно, для некоторого $N \in \NN$ выполнено $|a_n - a_m| < 1$ при $n, m > N$. Отсюда
    \begin{equation*}
        |a_n| = |a_n - a_{N + 1} + a_{N + 1}| \leq |a_n - a_{N+1}| + |a_{N+1}| < 1 + |a_{N + 1}|
    \end{equation*}
    при $n > N$. Значит,
    \begin{equation*}
        |a_n| \leq M = max \{1 + |a_{N+1}, |a_1|, \dots, |a_N|\}.
    \end{equation*}
    У ограниченной последовательности $\{a_n\}^{\infty}_{n=1}$ по теореме Больцано есть хотя бы один частичный предел $a$. Пусть к нему сходится подпоследовательность $\{a_{n_k}\}_{k=1}^{\infty}$, т.е. для каждого $\eps > 0$ найдется такой номер $k_0$, что при $k > k_0 \ |a - a_{n_k}| < \eps$. Кроме того, в силу фундаментальности найдется номер $N$, для которого $a_n - a_m < \eps$ при $n, m > N$. Пусть $k$ выбрано так, что $k > k_0$ и $n_k > N$, тогда при каждом $n > N$ выполнено, что $|a - a_n| \leq |a - a_{n_k}| + |a_{n_k} - a_n| < 2 \eps$.
    \end{proof}
    \subsection{Вычисление $\sqrt{2}$ с помощью рекуррентной формулы $a_{n+1} = 1 + \frac{1}{1 + a_n}$}, обоснование сходимости. [3.29]
    Пусть $a_{n+1} = 1 + \frac{1}{1 + a_n}, a_1 = 1$ Докажите, что $a_n$ сходится, и найдите предел. 
    \begin{proof}
    Заметим, что $a_n \geq 1$ и 
    \begin{equation*}
        |a_{n+1} - a_n| = \left|\frac{1}{1 + a_n} - \frac{1}{1 + a_{n-1}}\right| = \frac{|a_n - a_{n-1}|}{(1 + a_n)(1 + a_{n-1})} \leq \frac{1}{4} |a_n - a_{n-1}| \leq \left(\frac{1}{4}\right)^{n-1} |a_2 - a_1| = \left(\frac{1}{4}\right)^{n-1} \frac{1}{2}.
    \end{equation*}
    Отсюда при $m > n$
    \begin{equation*}
        |a_m - a_n| \leq |a_m - a_{m-1}| + \dots + |a_{n + 1} - a_n| \leq \frac{1}{2}\left(\left(\frac{1}{4}\right)^{m-2} + \dots + \left(\frac{1}{4}\right)^{n-1}\right) = 
    \end{equation*}
    \begin{equation*}
        = \frac{1}{2} \left(\frac{1}{4}\right)^{n-1} \frac{1 - (\frac{1}{4})^{m-n}}{1 - \frac{1}{4}} \leq \frac{8}{3} \left(\frac{1}{4}\right)^n.
    \end{equation*}
    Т.к. $(\frac{1}{4})^n \to 0$, то для каждого $\eps > 0$ найдется номер $N$, для которого $(\frac{1}{4})^n < \eps$ при $n > N$. Тем самым, для последовательности $\{a_n\}^{\infty}_{n=1}$ выполнен критерий Коши, а значит существует $a = \lim_{n \to \infty} a_n$. По арифметике предела, число $a$ удовлетворяет уравнению
    \begin{equation*}
        a(1 + a) = 1 + a + 1 \Longleftrightarrow a^2 = 2 \Longleftrightarrow a = \sqrt{2}
    \end{equation*}
    \end{proof}
    \subsection{Числовые ряды [3.31]. Критерий Коши для числовых рядов [3.33]. Необходимое условие сходимости числового ряда [3.34]. Расходимость гармонического ряда [3.35].}
    \subsubsection{Числовые ряды}
     Пусть $\{a_n\}^{\infty}_{n=1}$ --- числовая последовательность. \textbf{Числовым рядом} с членами $a_n$ называется выражение 
    \begin{equation*}
        a_1 + a_2 + a_3 + \dots = \sum_{k=1}^{\infty} a_k
    \end{equation*}
    Конечные суммы $S_n := \sum_{k=1}^{n} a_k$ называют \textbf{частичными суммами} ряда $\sum_{k=1}^{\infty} a_k$.

    Говорят, что ряд $\sum_{k=1}^{\infty} a_k$ \textbf{сходится}, если у последовательности $\{S_n\}^{\infty}_{n=1}$ существует предел, который называют суммой ряда. Если такого предела не существует, то говорят, что ряд не сходится или \textbf{расходится}.

    В силу арифметики предела на сходимость ряда (но не на сумму ряда) не влияет добавление (или отбрасывание) первых нескольких слагаемых.

    Также часто бывает удобно индексировать суммирование не только натуральным рядом. Под выражением $\sum_{k=k_0}^{\infty} a_k$ естественным образом подразумевается $\sum_{k=1}^{\infty} a_{k_0 + k-1}$ т.е. такой числовой ряд, чья $n$-я частичная сумма имеет вид $S_n = a_{k_0} + a_{k_0 + 1} + \dots + a_{k_0 + n -1}$
    \subsubsection{Критерий Коши для числовых рядов}
    Ряд $\sum_{k=1}^{\infty}$ сходится тогда и только тогда, когла для каждого $\eps > 0$ найдется такой номер $N$, что для всех $n > m > N$ выполнено $\left|\sum_{k = m + 1}^{n} a_k\right| = |S_n - S_m| < \eps$
    \subsubsection{Необходимое условие сходимости числового ряда}
    Если ряд $\sum_{k=1}^{n} a_k$ сходится, то $a_k \to 0$ при $k \to \infty$.
    \begin{proof}
    Действительно, из критерия Коши следует, что для каждого $\eps > 0$ найдется пример $N$, для которого при каждом $n > N + 1$ выполнено $|a_n| = |S_n - S_{n-1}| < \eps$.
    \end{proof}
    \subsubsection{Расходимость гармонического ряда}
    Докажите, что гармонический ряд $\sum_{k=1}^{\infty} \frac{1}{k}$ расходится.
    \begin{proof}
    Рассмотрим ряд $\sum_{k=1}^{\infty} \frac{1}{k}$. Исследуем его на выполнение условия критерия Коши: пусть $n > m$, тогда 
    \begin{equation*}
        \left|\sum_{k=m+1}^{n} \frac{1}{k}\right| \geq \frac{n - m}{n}.
    \end{equation*}
    Какой бы теперь ни был задан номер $N$ всегда можно взять $m > N$ (например $m = N + 1$) и $n = 2m$, тогда $\left|\sum_{k=m+1}^{n} a_k\right| \geq \frac{1}{2}$, а значит условие критерия Коши не выполнено и ряд расходится.
    \end{proof}
    \subsection{Абсолютная и условная сходимость рядов [3.37]. Сходимость абсолютно сходящегося ряда [3.36]. Ограниченность частичных сумм сходящегося ряда с неотрицательными членами [3.39]. Признак сравнения [3.40].}
    \subsubsection{Абсолютная и условная сходимость}
    Говорят, что ряд $\sum_{k=1}^{\infty} a_k$ \textbf{сходится абсолютно}, если сходится ряд $\sum_{k=1}^{\infty} |a_k|$.

    Говорят, что ряд $\sum_{k=1}^{\infty} a_k$ \textbf{сходится условно}, если он сходится, а ряд $\sum_{k=1}^{\infty} |a_k|$ расходится.
    \subsubsection{Сходимость абсолютно сходящегося ряда}
    Из сходимости ряда $\sum_{k=1}^{\infty} |a_k|$ следует сходимость ряда $\sum_{k=1}^{\infty} a_k$.
    \begin{proof}
        Действительно, из сходимости ряда $\sum_{k=1}^{\infty} |a_k|$ следует выполнение для него условия критерия Коши, а именно, для каждого $\eps > 0$ найдется такое натуральное число $N$, что при $n > m > N$ выполнено $\sum_{k=m+1}^{n} |a_k| < \eps$. Т.к. $\left|\sum_{k=m+1}^{n} a_k\right| \leq \sum_{k=m+1}^{n} |a_k|$, то условие критерия Коши выполнено и для ряда $\sum_{k=1}^{\infty} a_k$, а значит он сходится.
    \end{proof}
    \subsubsection{Ограниченность частичных сумм сходящегося ряда с неотрицательными членами}
    Пусть $a_k \geq 0$, тогда ряд $\sum_{k=1}^{\infty}$ сходится тогда и только тогда, когда последовательность его частичных сумм ограничена. 
    \begin{proof}
    Утверждение следует из того, что последовательность частичных сумм не убывает.
    \end{proof}
    \subsubsection{Признак сравнения}
    Пусть $0 \leq a_n \leq b_n$. Если ряд $\sum_{k=1}^{\infty} b_k$ сходится, то сходится и ряд $\sum_{k=1}^{\infty}$. Наоборот, если ряд $\sum_{k=1}^{\infty} a_k$ расходится, то расходится и ряд $\sum_{k=1}^{\infty} b_k$
    \subsection{Признак Лобачевеского-Коши [3.41]. Сходимость и расходимость рядов $\sum_{n=1}^{\infty} \frac{1}{n^p}$} в зависимости от параметра $p$ [3.42].
    \subsubsection{Признак Лобачевеского-Коши}
    Пусть $\{a_n\}_{n=1}^{\infty}$ --- невозрастающая последовательность, $a_n \geq 0$. Ряд $\sum_{k=1}^{\infty} a_k$ сходится тогда и только тогда, когда сходится ряд $\sum_{k=1}^{\infty} 2^k a_{2^k}$.
    \begin{proof}
    Заметим, что 
    \begin{equation*}
        a_2 + 2 a_4 + 4 a_8 + \dots + 2^{n-1} a_{2^n} \leq a_2 + a_3 + a_4 + a-5 + a_6 + a_7 + a_8 + \dots + a_{2^n} \leq 2 a_2 + 4 a_4 + 8 a_8 + \dots + 2^n a_{2^n}.
    \end{equation*}
    Отсюда получаем, что из ограниченности частичных сумм ряда $\sum_{k=1}^{\infty} 2^k a_{2^k}$ следует ограниченность частичных сумм ряда $\sum_{k=1}^{\infty} a_k$ и наоборот.
    \end{proof}
    \subsubsection{Сходимость и расходимость $p$-рядов}
    Ряды $\sum_{k=1}^{\infty} \frac{1}{k^p}$ сходится при $p > 1$ и расходится при $p \leq 1$
    \begin{proof}
        Во-первых, при $p \leq 0$ слагаемое $\frac{1}{k^p}$ не стремится к нулю. а значит ряд не сходится. Теперь рассмотрим $p > 0$. По доказанному признаку Лобачевеского-Коши ряд $\sum_{k=1}^{\infty} \frac{1}{k^p}$ сходится тогда и только тогда, когда сходится ряд
        \begin{equation*}
            \sum_{k=1}^{\infty} \frac{2^k}{2^{kp}}=\sum_{k=1}^{\infty} (2^{1-p})^k
        \end{equation*}
        Это геометрическая прогрессия, которая сходится при $2^{1-p} < 1$, т.е. при $p > 1$.
    \end{proof}
    \subsection{Перестановки слагаемых в рядах [3.43]. Теорема о независимости суммы абсолютно сходящегося ряда от порядка слагаемых [3.44]. Теорема Римана ([3.45], без док-ва)}
    \subsubsection{Перестановки слагаемых в рядах}
    Будем говорить, что ряд $\sum_{j=1}^{\infty} \widetilde{a}_{j}$ получен перестановкой членов ряда $\sum_{k=1}^{\infty} a_k$, если существует последовательность натуральных чисел $\{k_j\}^{\infty}_{j=1}$, задающая взаимно однозначное преобразование множества $\NN$, и такая, что $\forall j \in \NN \ \widetilde{a}_j = a_{k_j}$
    \subsubsection{Теорема о независимости суммы абсолютно сходящегося ряда от порядка слагаемых.}
    Если ряд $\sum_{j=1}^{\infty} \widetilde{a_j}$ получен перестановкой членов абсолютно сходящегося ряда $\sum_{k=1}^{\infty} a_k$, то ряд $\sum_{j=1}^{\infty} \widetilde{a_j}$ абсолютно сходится и его сумма равна сумме ряда $\sum_{k=1}^{\infty} a_k$.
    \begin{proof}
        Доказательство теоремы разобьем на два шага.

        \textbf{Шаг 1.} Рассмотрим случай, когда $a_k \geq 0 \ \ \forall k \in \NN$. Для любого $n \in \NN$ определим
        \begin{equation*}
            M_n = \underset{j \in 1, n}{\max} \ k_j
        \end{equation*}
        Тогда для любого $n \in \NN$ имеем
        \begin{equation*}
            \sum_{j=1}^{n} \widetilde{a_j} = \sum_{j=1}^{n} a_k \leq \sum_{k=1}^{M_n} a_k \leq \sum_{k=1}^{\infty} a_k
        \end{equation*}
        Переходя к пределу при $n \to \infty$, получаем $\sum_{j=1}^{\infty} \leq \sum_{k=1}^{\infty} a_k$. Поскольку ряд $\sum_{k=1}^{\infty} a_k$ может быть получен из ряда $\sum_{j=1}^{\infty} \widetilde{a_j}$ обратной перестановкой слагаемых, то $\sum_{k=1}^{\infty} a_k \leq \sum_{j=1}^{\infty} \widetilde{a_j}$. Поэтому при перестановке слагаемых ряда с неотрицательными членами его сумма не меняется.

    \textbf{Шаг 2.} Рассмотрим общий случай. Применяя утверждение шага 1 для сходящего ряда $\sum_{k=1}^{\infty} |a_k|$, получаем сходимость ряда $\sum_{j=1}^{\infty} |\widetilde{a_j}|$. ПОэтому ряд $\sum_{j=1}^{\infty} \widetilde{a_j}$ сходится абсолютно. Для любого $k \in \NN$ обозначим через
    \begin{equation*}
        a_k^+ = \max \{a_k, 0\}, \ \ a_k^- = \max \{-a_k, 0\}
    \end{equation*}
    Тогда при всех $k \in \NN$
    \begin{equation*}
        a_k = a_k^+ - a_k^-, \ |a_k| = a_k^+ + a_k^-,
    \end{equation*}
    \begin{equation*}
        0 \leq a_k^+ \leq |a_k|, \ 0 \leq a_k^- \leq |a_k|
    \end{equation*}
    В силу утверждения, доказанного на шаге 1,
    \begin{equation*}
        \sum_{j=1}^{\infty} a_{k_j}^+ = \sum_{k=1}^{\infty} a_k^+, \ \sum_{j=1}^{\infty} a_{k_j}^- = \sum_{k=1}^{\infty} a_k^-
    \end{equation*}
    причем эти ряды сходятся по признаку сравнения. Поэтому
    \begin{equation*}
        \sum_{j=1}^{\infty} \widetilde{a_j} = \sum_{j=1}^{\infty} a_{k_j} = \sum_{j=1}^{\infty} \left(a_{k_j}^+ - a_{k_j}^-\right) = \sum_{j=1}^{\infty} a_{k_j}^+ - \sum_{j=1}^{\infty} a_{k_j}^- - \sum_{k=1}^{\infty} a_k^- = \sum_{k=1}^{\infty} a_k.
    \end{equation*}
    \subsubsection{Теорема Римана}
    Если ряд $\sum_{k=1}^{\infty} a_k$ сходится условно, то для любого $x \in \RR \cup \{+\infty, -\infty\}$ можно так переставить члены ряда $\sum_{k=1}^{\infty} a_k$, что полученный ряд $\sum_{j=1}^{\infty} \widetilde{a}_j$ будет иметь сумму, равную $x$.
\end{proof}
\subsection{Определения предела функции (по множеству) по Коши [4.6] и по Гейне [4.5], их эквивалентность [4.10]}
\subsubsection{Определение предела функции по Коши}
Пусть функция $f$ на некотором множестве $D \subset \RR$ и пусть $a$ предельная для $D$ точка. Число $A$ называется пределом функции $f$ в точке $a$ (по множеству $D$), если для каждого $\eps > 0$ найдется такое $\delta > 0$, что $|f(x) - A| < \eps$ для каждого $x \in D \cap B_{\delta}'(a)$. Используют обозначения $\lim_{x \to a} f(x) = A$ или $f(x) \to A$ при $x \to a$.

    т.е. $\lim_{x \to a} f(x)= A$, если 
    \begin{equation*}
        \forall \eps > 0 \exists \delta > 0 : \forall x \in D \cap B_{\delta}'(a) \ \ |f(x) - A| < \eps
    \end{equation*}
    или
    \begin{equation*}
        \forall \eps > 0 \exists \delta > 0 : \forall x \in D, 0 < |x - a| < \delta \ \ |f(x) - A| < \eps
    \end{equation*}
    \subsubsection{Определение предела функции по Гейне}
    Пусть функция $f$ определена на некотором множестве $D \subset \RR$ и пусть $a$ предельная для $D$ точка. Число $A$ называется пределом фнукции $f$ в точке $a$ по множеству $D$, если для каждой последовательности точек $x_n \in D \setminus \{a\}, x_n \to a,$ выполнено $f(x_n) \to A$ при $n \to \infty$.
    \subsubsection{Эквивалентность пределов по Коши и по Гейне}
    Определения предела функции по Коши и по Гейне эквивалентны.
    \begin{proof}
    Пусть $\lim_{x \to a} f(x) = A$ в смысле Коши. Рассмотрим последовательность точек $x_n \in D \setminus \{a\}$, сходящуюся к точке $a$. Для каждого $\eps > 0$ надйется $\delta > 0$, для которого $|f(x) - A| < \eps$ при $x \in D \cap B_{\delta}'(a).$ Найдется номер $N$, для которого $x_n \in B_{\delta}(a)$ при $n > N$. Т.к. при $n > N$ точки $x_n \in D \setminus \{a\}$ и $x_n \in B_{\delta}(a)$, то при $n > N$ выполнено $|f(x_n) - A| < \eps$. Это и означает, что $f(x_n) \to A$ при $n \to \infty$. Таким образом, число $A$ является пределом функции $f$ в точке $a$ в смысле Гейне.

    Пусть число $A$ не является пределом функции $f$ в точке $a$ в смысле Коши. Это означает, что нашлось такое $\eps > 0$, что для каждого $\delta > 0$ есть точка $x_{\delta} \in D \cap B_{\delta}'(a)$, для которой $|f(x_{\delta}) - A| \geq \eps$. Для последовательности точек $x_{\frac{1}{n}} \in D \setminus \{a\}$ выполнено $x_{\frac{1}{n}} \to a$, но последовательность точек $f(x_{\frac{1}{n}})$ не сходится к $A$. Таким образом, число $A$ не является пределом функции $f$ в точке $a$ в смысле Гейне.
    \end{proof}
    \subsection{Единственность и арифметические свойства предела функции: линейность, предел произведения и отношения [4.11]}
    \subsubsection{Единственность предела}
    если $\lim_{x \to a} f(x) = A$ и $\lim_{x \to a} f(x) = B$, то $A = B$
    \subsubsection{Линейность предела}
    если $\lim_{x \to a} f(x) = A$ и $\lim_{x \to a} g(x) = B$, то 
    \begin{equation*}
        \lim_{x \to a}(\alpha f(x) + \beta g(x)) = \alpha A + \beta B \ \ \forall \alpha, \beta \in \RR
    \end{equation*}
    \subsubsection{Предел произведения}
     если $\lim_{x \to a} f(x) = A$ и $\lim_{x \to a} g(x) = B$, то $\lim_{x \to a} (f(x) \cdot g(x)) = A \cdot B$
    \subsubsection{Предел отношения}
     если $\lim_{x \to a} f(x) = A, \lim_{x \to a} g(x) = B \ne 0$ и $g(x) \ne 0$ при $x \in D$, то $\lim_{x \to a} \frac{f(x)}{g(x)} = \frac{A}{B}$
    \subsubsection{Доказательство свойств}
    \begin{proof}
    Свойства  следуют из аналогичных свойств для предела последовательности и определения предела функции по Гейне.
    \end{proof}
    \subsection{Предел и неравенства, ограниченность, отделимость [4.12]. Теорема о пределе сложной функции [4.13].}
    \subsubsection{Предел функции и неравенства}
    Пусть функции $f, g, h$ определены на некотором множестве $D \subset \RR$ и пусть $a$ предельная для $D$ точка. Тогда
    \begin{enumerate}
        \item (переход к пределу в неравенстве) если $\exists r > 0: \ \ f(x) \leq g(x)$ при $x \in D \cap B_{r}'(a)$ и $\lim_{x \to a} f(x) = A, lim_{x \to a} g(x) = B$, то $A \leq B$
        \item (предел зажатой функции) если $\exists r > 0: \ \ f(x) \leq h(x) \leq g(x)$ при $x \in D \cap B_{r}'(a)$ и  $\lim_{x \to a} f(x) = lim_{x \to a} g(x) = A$, то $\lim_{x \to a} h(x) = A$
        \item (ограниченность функции) если $\lim_{x \to a} f(x) = A$, то найдутся такие $\delta > 0$ и $C > 0$, что $|f(x)| \leq C$ при каждом $x \in D \cap B_{\delta}'(a)$
        \item (отделимость от нуля) если $\lim_{x \to a} f(x) = A \ne 0$, то найдется такое $\delta > 0$, что $|f(x)| > \frac{|A|}{2}$ при $x \in D \cap B_{\delta}'(a)$.
    \end{enumerate}
    \begin{proof}
    Свойства 1) - 2) следует из аналогичных свойств для предела последовательности и предела функции по Гейне.

    3) Найдется такое $\delta > 0$, что $|f(x) - A| < 1$ при $x \in D \cap B_{\delta}'(a)$. Таким образом, при $x \in D \cap B_{\delta}'(a)$ выполнено $|f(x)| < 1 + |A|$.

    4) Найдется такое $\delta > 0$, что $|f(x) - A| < \frac{|A|}{2}$ при $x \ in D \cap B_{\delta}'(a)$. Таким образом, при $x \in D \cap B_{\delta}'(a)$ выполнено $|f(x)| > \frac{|A|}{2}$
    \end{proof}
    \subsubsection{Теорема о пределе сложной функции}
    Пусть $f \colon D \to E, g \colon E \to \RR, a$ -- предельная точка множества $D, b$ -- предельная точка множества $E, \lim_{x \to a} f(x) = b, \lim_{y \to b} g(y) = c$ и есть такая проколотая окрестность $B_{\delta}' (a)$ точки $a$, что $f(x) \ne b$ для каждой точки $x \in B_{\delta}'(a) \cap D$. Тогда $\lim_{x \to a} g(f(x)) = c$
    \begin{proof}
    Пусть $x_n \to a, x_n \in D, x_n \ne a$ Т.к. $f(x) \ne b$ для каждой точки $x \in B_{\delta}'(a) \cap D$, то найдется такой номер $N$, что $f(x_n) \ne b$ при $n > N$. Поэтому последовательность $f(x_{N + 1}), f(x_{N + 2}), \dots$ состоит из элементов множества $E$, ни один из этих элементов не совпадает с $b$ и эта последовательность сходится к $b$. Поэтому последовательность $g(f(x_{N + 1})), g(f(x_{N + 2})), \dots$ сходится к $c$. Значит и всяс последовательность $\{g(f(x_n))\}$ сходится к $c$.
    \end{proof}
    \subsection{Замечательные пределы [4.14], [4.15]}
    \subsubsection{Первый замечательный предел}
    $\lim_{x \to 0} \frac{\sin x}{x} = 1$
    \begin{proof}
    Действительно, при $x \in (0, \frac{\pi}{2})$, сравнивая площади сектора с площадями двух треугольников, получаем 
    \begin{equation*}
        \frac{1}{2} \cdot 1 \cdot 1 \cdot \sin{x} \leq \frac{1}{2} \cdot 1^2 \cdot x \leq \frac{1}{2} \cdot 1 \cdot \tg{x}
    \end{equation*}
    откуда, в силу четности, при $x \in (-\frac{\pi}{2}, \frac{\pi}{2}), x \ne 0$, выполнено
    \begin{equation*}
        \cos{x} \leq \frac{\sin{x}}{x} \leq 1
    \end{equation*}
    Утверждение теперь следует из теоремы о пределе зажатой функции, т.к. $\lim_{x \to y} \cos{x} = \cos{y}$. Действительно, 
    \begin{equation*}
        |\cos{x} - \cos{y}| = 2 \left|\sin{\left(\frac{x + y}{2}\right)} \sin{\left(\frac{x - y}{2}\right)}\right| \leq 2 \left|\sin{\left(\frac{x - y}{2}\right)}\right| \leq |x - y|
    \end{equation*}
    \end{proof}
    \subsubsection{Второй замечательный предел}
    $\lim_{x \to + \infty} \left(1 + \frac{1}{x}\right)^x = e$
    \begin{proof}
    Пусть $f(x) := \left(1 + \frac{1}{[x] + 1}\right)^{[x]}$ и $g(x) := \left(1 + \frac{1}{[x]}\right)^{[x] + 1}$. Тогда $f(x) \leq (1 + \frac{1}{x})^x \leq g(x).$ Кроме того, т.к. $\lim_{n \to + \infty} \left(1 + \frac{1}{n + 1}\right)^n = \lim_{n \to + \infty} \left(1 + \frac{1}{n}\right)^{n + 1} = e$, то $\lim_{x \to + \infty} f(x) = \lim_{x \to + \infty} g(x) = e$. Утверждение теперь следует из теоремы о пределе зажатой функции. 
    \end{proof}
    \subsection{Критерий Коши [4.16]}
    Пусть $f \colon D \to \RR$ и $a$ - предельная точка $D$. Предел $\lim_{x \to a} f(x)$ существует тогда и только тогда когда для каждого $\eps > 0$ найдется такое $\delta > 0$, что для каждых $x, y \in B_{\delta}'(a) \cap D$ выполнено $|f(x) - f(y)| < \eps$
    \begin{proof}
    Если $\lim_{x \to a} f(x) = A$, то для каждого $\eps > 0$ найдется такое $\delta > 0$, что для произвольной точки $x \in B_{\delta}'(a) \cap D$ выполнено $|f(x) - A| < \frac{\eps}{2}$. Тогда для проивзольных точек $x, y \in B_{\delta}'(a) \cap D$ выполнено $|f(x) - f(y)| \leq |f(x) - A| + |A - f(y)| < \eps.$

    Предположим, что выполнено условие Коши. Тогда для произвольной последовательности точек $x_n \in D \setminus \{a\}, x_n \to a$, последовательность $\{f(x_n)\}$ является фундаменталньой, а значит сходится. Пусть $\lim_{n \to \infty} f(x_n) = A$. Если есть другая последовательность точек $y_n \in D \setminus \{a\}, y_n \to a$, то рассмотрим новую последовательность $z_{2k - 1} = x_k, z_{2k} = y_k$, т.е. это последовательность вида $x_1, y_1, x_2, y_2, \dots \subset D \setminus \{a\}$. Эта последовательность также сходится к $a$, поэтому последовательность образов $f(x_1), f(y_1), f(x_2), f(y_2), \dots$ снова оказывается фундаменталньой, а потому сходится. В силу того, что предел подпоследовательности сходящейся последовательности совпадает с пределом всем последовательности, получаем, что $\lim_{n \to \infty} f(y_n) = A$. Таким образом, что доказано существование предела по Гейне.
    \end{proof}
    \subsection{Односторонние пределы и теорема Вейерштрасса о существовании односторонних пределов монотонной ограниченной фуннкции [4.17], [4.18]}
    \subsubsection{Односторонние пределы}
    Пусть здесь и далее $D^{+}_a := D \cap (a, + \infty)$ и $D^{-}_a := D \cap (-\infty, a)$

    Пусть точка $a$ -- предельная для множества $D^{+}_a$ и существует предел функции $f$ по множеству $D^{+}_a$ в точке $a$. Этот предел называют пределом справа фнукции $f$ в точке $a$ и обозначают $\lim_{x \to a + 0} f(x)$ или просто $f(a + 0)$. Аналогично определяется предел слева, который обозначают $\lim_{x \ to a - 0} f(x)$ или $f(a - 0)$
    \subsubsection{Теорема Вейерштрасса}
    Пусть $f$ -- не убывает и ограничена на множестве $D, a$ -- предельная точка $D^{-}_a$. Тогде существует предел слева
    \begin{equation*}
        \lim_{x \to a - 0} f(x) = \sup \{f(x) \colon x \in D^{-}_a\}
    \end{equation*}
    Пусть $f$ - не убывает и ограничена на множестве $D, a$ -- предельная точка множества $D^{+}_a$. Тогда существует предел справа
    \begin{equation*}
        \lim_{x \to a + 0} f(x) = \inf \{f(x) \colon x \in D^{+}_a\}.
    \end{equation*}
    Аналогичные утверждения с заменой $\inf$ на $\sup$ справедлины и для невозрастающей функции.
    \begin{proof}
    Пусть $M = \sup \{f(x) : x \in D_a^-\}$. Тогда для каждого $\eps > 0$ найдется такая точка $x_n \in D_a^-$, что $M - \eps < f(x_0)$. Т.к. $f$ не убывает на $D_a^-$, то для каждого $x \in (x_0, a) \cap D_a^-$ выполнено $M - \eps < f(x_0) \leq f(x) \leq M < M + \eps$. Тогда, взяв $\delta := a - x_0$ получаем. что для каждого $x \in B_{\delta}'(a) \cap D_{a}^-$ выполнено $|f(x) - M| < \eps.$
    \end{proof}
    \subsection{Отношение эквивалентности функций, примеры эквивалентных функций, лемам о эквивалентности произведения и композии функций, предел эквивалентных функций. [4.24], [4.25], [4.26], [4.27], [4.29]}
    \subsubsection{Отношение эквивалентности функций}
     Пусть функции $f$ и $g$ определены и не обращаются в 0 в некоторой $B'(x_0).$ Функции $f$ и $g$ называются эквивалентными (пишут: $f(x) \sim g(x)$) \ \ при $x \to x_0$, если \ \ $\lim_{x \to x_0} \frac{f(x)}{g(x)} = 1.$
    \subsubsection{Лемма 4.25}
    Отношение эквивалентности функций при $x \to x_0$ является отношением эквивалентности на множестве функций, определенных и не обращающихся в ноль в $B_{\delta}'(x_0)$
    \begin{proof}
    Непосредственно из определения следует, что отношение $\sim$ обладает свойством рефлексивности. Пусть функции $f, g, h$ определены и не обращаются в $0$ в $B_{\delta}'(x_0)$. Если $f(x) \sim g(x)$ при $x \to x_0$, то по теореме о пределе частного
    \begin{equation*}
        \lim_{x \to x_0} \frac{g(x)}{f(x)} = \lim_{x \to x_0} \frac{1}{\frac{f(x)}{g(x)}} = \frac{1}{\displaystyle \lim_{x \to x_0} \frac{f(x)}{g(x)}} = 1
    \end{equation*}
    Поэтому $f(x) \sim h(x)$ при $x \to x_0$, т.е. отношение $\sim$ обладает свойством транзитивности.
    \end{proof}
    \subsubsection{Лемма 4.26}
    Пусть функции $f_1(x), f_2(x), g_1(x), g_2(x)$ определены и не обращаются в 0 в некоторой $b_{\delta}'(x_0)$ и пусть $f_1(x) \sim f_2(x), g_1(x) \sim g_2(x)$ при $x \to x_0$. Тогда $f_1(x) g_1(x) \sim f_2(x) g_2(x), \frac{f_1(x)}{g_1(x)} \sim \frac{f_2(x)}{g_2(x)}$ при $x \to x_0$.
    \begin{proof}
    Доказательство следует из теорем о пределе произведения в пределе частного.
    \end{proof}
    \subsubsection{Лемма 4.27}
    Если $f(x) \sim g(x)$ при $x \to x_0$, то $\lim_{x \to x_0} f(x) = \lim_{x \to x_0} g(x)$. а если один из пределов не существует, то не существует и другой.
    \begin{proof}
    Если $\exists \lim_{x \to x_0} g(x) \in \RR$, то по теореме о пределе произвдения $\exists \lim_{x \to x_0} f(x) = \lim_{x \to x_0} \frac{f(x)}{g(x)} g(x) = \lim_{x \to x_0} g(x)$. Аналогично, если $\exists \lim_{x \to x_0} f(x) \in \RR$, то $\exists \lim_{x \to x_0} g(x) = \lim_{x \to x_0} f(x)$.
    \end{proof}
    \subsubsection{Лемма 4.29}
    Пусть $f(y) \sim g(y)$ при $y \to y_0$, и пусть $y(x) \to y_0$ при $x \to x_0$ и $y(x) \ne y_0$ при $x \in \overset{o}{U}_{\delta}(x_0).$. Тогда $f(y(x)) \sim g(y(x))$ при $x \to x_0$.
    \begin{proof}
    По теореме о пределе сложной функции иеем $\lim_{x \to x_0} \frac{f(y(x))}{g(y(x))} = \lim_{y \to y_0} \frac{f(y)}{g(y)} = 1.$
    \end{proof}
    \subsection{Понятие функции, бесконечно малой или ограниченной относительно другой ($\overline{o}$ и $\underline{O}$), теорема о связи отношения эквивалентности и представления функции с помощью $\overline(o)$, свойства $\overline{o}$ и $\underline{O}$ [4.31], [4.33], [4.32], [4.34]}
    \subsubsection{Бесконечно малая функция}
     Пусть функции $f$ и $g$ определены в $B'(x_0)$ и функция $g(x)$ не обращается в 0. Говорят, что функция $f$ является бесконечно малой относительно функции $g$ при $x \to x_0$ и пишут $f(x) = o(g(x))$ при $x \to x_0$, если $\lim_{x \to x_0} \frac{f(x)}{g(x)}=0$
    \subsubsection{Функция ограничена относительно другой}
    Пусть функции $f$ и $g$ определены в $B_{\delta}' (x_0)$. Говорят, что функция $f$ ограничена относительно функции $g$, и пишут $f(x)$ = $O(g(x))$ при $x \to x_0$, если 
    \begin{equation*}
        \exists C \in \RR : \forall x \in B_{\delta}' (x_0) \ \ |f(x)| \leq C|g(x)|
    \end{equation*}
    \subsubsection{Теорема о связи отношения эквивалентности и представления функции с помощью $o$}
    $f(x) \sim g(x)$ при $x \to x_0 \Leftrightarrow f(x) - g(x) = o(g(x))$ при $x \to x_0$.
    \begin{proof}
    $f(x) \sim g(x)$ при $x \to x_0$
    \begin{equation*}
        \Longleftrightarrow \lim_{x \to x_0} \frac{f(x)}{g(x)} = 1 \Longleftrightarrow \lim_{x \to x_0} \frac{f(x) - g(x)}{g(x)} = 0
    \end{equation*}
    \begin{equation*}
        \Longleftrightarrow f(x) - g(x) = o(g(x)) \ \text{при} \ x \to x_0.
    \end{equation*}
    \end{proof}
    \subsubsection{Следствие из этой теоремы}
    \begin{itemize}
        \item $\sin{x} = x + o(x)$
        \item $\arcsin{x} = x + o(x)$
        \item $e^x = 1 + x + o(x)$
        \item $\sh{x} = x + o(x)$
        \item $\tg{x} = x + o(x)$
        \item $\arctg{x} = x + o(x)$
        \item $\ln (1 + x) = x + o(x)$
        \item $\th{x} = x + o(x)$
    \end{itemize}
    при $x \to 0$
    \subsubsection{Свойства $o$ и $O$}
     \begin{enumerate}
        \item $o(f) \pm o(f) = o(f)$
        \item $O(f) \pm O(f) = O(f)$
        \item $o(f) = O(f)$
        \item $o(O(f)) = o(f)$
        \item $O(o(f)) = o(f)$
        \item $O(O(f)) = O(f)$
        \item $o(f) \cdot O(g) = o(fg)$
        \item $O(f) \cdot O(g) = O(fg)$
        \item $(o(f))^{\alpha} = o(f^{\alpha}) \ \forall \alpha > 0$
        \item $(O(f))^{\alpha} = O(f^{\alpha}) \ \forall \alpha > 0$
    \end{enumerate}
    \begin{proof}
    Докажем, например, первое утверждене. Требуется доказать, что если $g_1(x) = o(f(x)), g_2(x) = o(f(x))$ при $x \to x_0$, то $g_1(x) \pm g_2(x) = o(f(x))$ при $x \to x_0$. Действительно из условий $\lim_{x \to x_0} \frac{g_1(x)}{f(x)}=0, \lim_{x \to x_0} \frac{g_2(x)}{f(x)} = 0$ следует $\lim_{x \to x_0} \frac{g_1(x) \pm g_2(x)}{f(x)} = 0$, т.е. $g_1(x) \pm g_2(x) = o(f(x))$ при $x \to x_0$. Остальные утверждения проверяются аналогично.
    \end{proof}
\end{document}